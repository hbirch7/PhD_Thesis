\chapter{Conclusion}\label{chap:Conclusion}
Contributions of the author to the commissioning and operation of the LZ Outer Detector and the 2024 WIMP search analysis have been presented in this thesis.

The Outer Detector plays a plays a crucial role in suppressing backgrounds for the direct detection of WIMPs. The primary purpose of the Outer Detector is to detect neutrons which have coincident signals within the TPC. Neutrons which scatter in the TPC go on to capture on gadolinium and hydrogen or recoil of protons which produce energy depositions of $\sim100$~keV corresponding to pulses of light $\sim15$~photons in size. Outer Detector PMTs which detect these light signals require single photoelectron calibration to be sensitive to such low energy interactions. This work presents the method use to model the PMT response to single photoelectrons and the development of the optical calibration system used to perform single photon calibration of the Outer Detector PMTs. The single photoelectron sensitivity stability of the WS2024 of $(1.37\pm0.25)\%$ at an operating gain of $2\times10^6$ has been achieved through the ongoing efforts by the author over a five year period. Understanding the response of the Outer Detector PMTs and the stability of the single photoelectron signal during science runs is imperative to the performance of the veto detectors and their ability to effectively identify events which coincident signals in the TPC.

Improvements to the Outer Detector geometry in simulation are reflected in the good agreement in comparisons of measured neutron capture timing observed in calibration data sets. Accurate modelling of the detector response to neutrons in simulations is crucial for the optimisation of the veto coincidence selection for the WIMP search analysis. The optimisation of the veto coincidence selection prior to the 2024 WIMP search analysis resulted in an estimated neutron tagging efficiency of $(92\pm4)\%$ with a $3\%$ impact on detector live time. This result presents factor of two reduction in live time impact by the veto coincidence selection whilst maintaining the same tagging efficiency as measured in the WS2022 science run. The veto coincidence selection was additionally used to generate a sample data set, used to estimate the number of single scatter events which results from detector neutrons. An expectation of $0.0^{+0.2}$ single scatter events are attributed to detector neutrons using a likelihood fit analysis at 68\% confidence level.

In addition to neutrons produced through radioactive decays of detector materials, neutrons are produced through the interaction of atmospheric muons with the detector and surround materials. To understand the muon induced background of LZ, an initial muon flux measurement has been performed. A reconstructed flux of $(5.119 \pm 0.080 \textrm{( stat.)} \pm 0.049 \textrm{ (syst.)})\times10^{-9}$~cm$^{-2}$~s$^{-1}$ is derived from the ratio of muon rate found in data and simulation taking into account the muon flux used in the simulation. In addition to the muon flux measurement, the best fit diurnal muon modulation frequency $f=(1/21.7)^{+0.005}_{-0.007}~\text{day}^{-1}$ (90\% confidence level) is found with a local significance of $3.05\sigma$ with $p=0.001$. For longer period, the flux has no significant modulation. Future studies will investigate this measurement further to understand the effects of energy selection, depth and more.
Further efforts have been made to support the WIMP search analysis through the development of the muon veto and hold off. The veto was initially developed for the WS2022 science run and then optimised for the WS2024 science run. The muon veto has $1.9\%$ impact on the WS2024 detector live time. This work will have a lasting legacy in LZ as further studies will continue to quantify the muon induced backgrounds using the muon event selection developed for the flux measurement.

The 2024 WIMP search analysis which LZ used to report its world-leading result has been summarised. This result is the near beginning of LZ showcasing its ability to search for WIMPs and new phenomena. The experiment continues to collect data towards a target of 1000-day live time that will enable increased sensitivity for WIMPs. Not only is LZ the current world leader in WIMP direct detection dark matter, it additionally is being used to further understand the effects that accidentals and backgrounds have on WIMP sensitivity. This work will prove crucial when considering different design requirements for the next-generation liquid xenon dark matter observatory, XLZD.

\begin{flushright}
\textit{Yesterday is history, today is a gift, and tomorrow is a mystery.}

David Lama
\end{flushright}