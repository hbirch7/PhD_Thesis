\chapter{Dark matter overview}\label{chap:DarkMatterOverview}
\section{Evidence for dark matter}\label{sec:DMOverview/Evidence4DM}
In the latter part of the 19th century, astronomers proposed the existence of non-visible matter to explain the uneven distribution of stars in the sky \cite{HistoryofDM}.
Lord Kelvin made one of the first attempts to estimate the amount of dark bodies in the Milky Way in 1904.
Kelvin postulated that, if stars in the Milky Way can be described like particles in a gas, acting under the influence of gravity then a relationship can be formed between the size of the system and the velocity dispersion of its constituents \cite{Kelvin1904}.
The term "dark matter" ("matière obscure") was first coined by Henri Poincar\'e in 1906 whilst arguing that the amount of dark matter would be less than or equal to the amount of visible light after studying the velocity disperision predicitons made by Kelvin \cite{HPon}.
In 1922, Jacobus Kapteyn produced one of the earliest models which quantitatively described the size and shape of the Milky Way. 
His model treated the distribution of stars as a flattened disk rotating around axis orientated towards the galactic pole.
Kapteyn came to similar conclusions as Poincar inwhich he believed that the prescence of large amounts of unseen matter was unlikely \cite{Kapteyn1922}.
Although these early observations provided inconclusive evidence to towards the exsistence of dark matter, they did however provide a firm foundation on which later studies could establish solid reasoning to search for the missing mass in the Galaxy.

\subsection{Virial theorem and the Coma Cluster}\label{sec:DMOverview/ViralTheorem}

\subsection{Galaxy rotation curves}\label{sec:DMOverview/RotationCurves}

\subsection{Gravitational lensing}\label{sec:DMOverview/GravLens}

\subsubsection{The Bullet Cluster}\label{sec:DMOverview/BulletCluster}

\section{The Cosmic Microwave Background}\label{sec:DMOverview/CMB}

\subsection{$\Lambda$CDM model}\label{sec:DMOverview/LambdaCDM}

\section{Avoiding dark matter}\label{sec:DMOverview/AvoidDM}

\subsection{Modification of Newtonian dynamics}\label{sec:DMOverview/MOND}

\subsection{MACHOs}\label{sec:DMOverview/MACHOs}

\section{Possible candidates for dark matter}\label{sec:DMOverview/Candidates4DM}

\subsubsection{WIMPs}\label{sec:DMOverview/WIMPs}

\subsection{Sterile neutrinos}\label{sec:DMOverview/Neutrinos}

\subsection{Axions}\label{sec:DMOverview/Axions}

\section{Searching for dark matter}\label{sec:DMOverview/DetectionOfDM}

\subsection{Indirect detection of dark matter}\label{sec:DMOverview/IndirectDM}

\subsection{Dark matter production at colliders}\label{sec:DMOverview/DMProdColliders}

\subsection{Direct detection of dark matter}\label{sec:DMOverview/DirectDetection}

\section{Current status for dark matter searches}\label{sec:DMOverview/DMCurrentStatus}