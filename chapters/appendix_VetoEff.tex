\section{TPC based cuts used in the veto efficiency studies}\label{sec:app/WSCoreCuts}
A series of cuts have been developed for the WIMP search analysis to select candidate events based on a number of TPC detector parameters. The cuts can be broken down into four broad categories: Live time cuts, Physics cuts, S1 pulse based cuts, and S2 pulse based cuts. Each of the tables below provides a brief description of the different cuts in each of the categories.
\begin{table}[h!]
    \centering
    \caption{Live time cuts. This series of cuts a used to remove period of time with events that contain spurious combinations of pulses in the detector.}
    \begin{tabular}{|m{10em}m{22em}|}
    \hline
    \textbf{Name} & \textbf{Brief description} \\
    \hline\hline
    Burst noise & Removes events with a high concentration of noise in the TPC.\\
    \hline
    Muon hold off & Excludes a fixed period of time after a muon passed through the TPC.\\
    \hline
    Bad buffer cuts & Removes events for which the DAQ buffers were not fully operational.\\
    \hline
    Excess Area cut & Requires that the summed area of the pulses before the S1 and between the S1 and the S2 is lower than the S2 area. This is important to catch events with high sustained rates that other cuts missed.\\
    \hline
    \end{tabular}
\end{table}
\begin{table}[h!]
    \centering
    \caption{Physics cuts. These cuts define the search space where LZ will conduct the WIMP search in $\{\text{S1}c,\text{log}_{10}(\text{S2}c)\}$ space. The fiducial volume is the inner cylindrical volume of xenon of lower background rate.}
    \begin{tabular}{|m{10em}m{22em}|}
    \hline
    \textbf{Name}&\textbf{Brief description}\\
    \hline\hline
    S1 Threshold & Defines the search boundaries for the WS in S1 area. Selection window: $3<\text{S1}c<80~\text{phd}$ \\
    \hline
    S2 Threshold & Defines the search boundaries for the WS in S2 pulse areas. Selection window: $\text{S2}>645~\text{phd}$ and $\text{log}_{10}(\text{S2}c)<4.5$ \\
    \hline
    Fiducial volume & Defines an inner region of the detector where the wall background leakage in the WS ROI is below a certain threshold.\\
    \hline
    \end{tabular}
\end{table}
\begin{table}[h!]
    \centering
    \caption{S2 pulse cuts. In combination with S1 based cuts, this series of cuts target populations of isolated S2 pulses which lead to accidentals in the TPC. These cuts were developed in isolation of other cuts, each for a specific type of S1 or S2 shape, or event pathology \cite{mwilliams:thesis}.}
    \begin{tabular}{|m{10em}m{22em}|}
    \hline
    \textbf{Name}&\textbf{Brief description}\\
    \hline\hline
    Narrow S2 &Targets “flat-top” S2 pulses, which typically have a narrow width.\\
    \hline
    S2 rise time & Targets “flat-top” S2 pulses, which typically have a steep rise.\\
    \hline
    S2 early peak & Targets extraction region gas events with a characteristic “bump” at the beginning.\\
    \hline
    S2 XY quality & Rejects events with a poor TPC-$xy$ $\chi^2$ from reconstruction minimization.\\
    \hline
    S2 TBA & Targets events occurring above the anode.\\
    \hline
    S1 Stinger & Removes events with a small S1 pulse following an SE/S2/Other pulse.\\
    \hline
    S1 TBA vs drift time& Exploits the empirical relationship between S1 TBA and drift time to reject accidental events.\\
    \hline
    S1 HSC & Removes events that exhibit large maximum channel areas (i.e. most of the light are concentrated in a single PMT).\\
    \hline
    S1 shape & Looks for S1s with slightly distorted shapes, which are more characteristic of accidental S1 than real S1.\\
    \hline
    \end{tabular}
\end{table}
\begin{table}[h!]
    \centering
    \caption{S1 pulse cuts. In combination with S2 based cuts, this series of cuts target populations of isolated S1 pulses which lead to accidentals in the TPC. These cuts were developed in isolation of other cuts, each for a specific type of S1 or S2 shape, or event pathology \cite{mwilliams:thesis}.}
    \begin{tabular}{|m{10em}m{22em}|}
    \hline
    \textbf{Name}&\textbf{Brief description}\\
    \hline\hline
    S1 Stinger & Removes events with a small S1 pulse following an SE/S2/Other pulse.\\
    \hline
    S1 TBA & Exploits the empirical relationship between S1 TBA and drift time to reject accidental events.\\
    \hline
    S1 HSC & Removes events that exhibit large maximum channel areas (i.e. most of the light are concentrated in a single PMT).\\
    \hline
    S1 shape & Looks for S1s with slightly distorted shapes, which are more characteristic of accidental S1 than real S1.\\
    \hline
    \end{tabular}
\end{table}