\chapter{Introduction}
Through cosmological observations it has been shown that there is five times more dark matter than regular matter in our Universe, however the nature of dark matter (DM) remains unknown. This is one of the big open questions physicists are yet to answer. One favoured candidate is a particle called Weakly Interacting Massive Particle (WIMP). The LZ experiment is currently the leading dark matter direct detection experiment in search for WIMPs. The detector is located on the 4,850~ft level (4,300~m~w.e.) of the Sanford Underground Research Facility (SURF) in the Homestake Mine (Lead, SD) \cite{LZNIMA}. At the core of the experiment is a dual-phase Time Projection Chamber (TPC) which is sensitive to low energy nuclear recoils (NR), the signal which is produced through WIMPs interacting with liquid noble gases. One of the main backgrounds in a WIMP search are neutrons as they also interact through nuclear recoils and thus LZ employs an active veto system to remove them. Theoretically WIMPs will only interact with the xenon target however neutrons would interact in both the TPC and veto detectors.

The TPC is housed within a vacuum insulated cyrostat with a layer of liquid xenon (Skin) which acts as high voltage stand-off, this region is also instrumented with PMTs and is part of the active veto system. The LXe Skin is used to veto mostly gamma ray interactions within the TPC volume, also being sensitive to neutrons. The cryostat is surrounded near hermetically by ten acrylic vessels filled with  Gadolinium loaded liquid scintillator (GdLS). The GdLS is observed by 120 PMTs and stands within 238t of DI water which provides shielding to the detector and additional target material to veto atmospheric muons.

This thesis presents the work done to ...

\autoref{chap:DarkMatterOverview} describes...

\autoref{chap:DarkMatterOverview} introduces the LZ dark matter experiment...

\autoref{chap:ODCommissioning} and onwards focuses on the work undertaken for this thesis...

\autoref{chap:VetoEfficiency} describes the work to...

\autoref{chap:Muons} describes the work to...