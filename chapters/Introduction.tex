\chapter{Introduction}
Through cosmological observations, it has been shown that there is five times more dark matter than regular matter in our Universe; however, the nature of dark matter remains unknown. This is one of the big open questions physicists are yet to answer. Dark matter is observed and defined by its gravitational effects on astrophysical structures and its role in the evolution of the Universe as we know it today. It is assumed that dark matter consists of a new fundamental particle (or family of particles) and that it is weakly coupled to the Standard Model. One favoured candidate is a particle called a Weakly Interacting Massive Particle (WIMP) that could be detected through interactions with matter. A review of dark matter, observational evidence to support its existence, various candidates, and possible detection mechanisms and experiments is presented in \autoref{chap:DarkMatterOverview}.

The LUX-ZEPLIN (LZ) experiment is currently the leading dark matter direct detection experiment in search for WIMPs \cite{LZCollaboration:2024lux}. The detector is located on the 4,850~ft level of the Sanford Underground Research Facility in the Homestake Mine (Lead, SD) \cite{LZNIMA}. At the core of the experiment is a dual-phase Time Projection Chamber (TPC), which is sensitive to low-energy nuclear recoils, the signal which is produced through WIMPs interacting with liquid noble gases. One of the main backgrounds in a WIMP search is neutrons, as they also interact through nuclear recoils, and thus LZ employs an active veto system to remove them. WIMPs will only interact with the xenon target due to the theoretically predicted interaction cross-section of the WIMP; however, neutrons interact in both the TPC and veto detectors. The TPC is housed within a vacuum-insulated cryostat with a layer of liquid xenon (Skin), which is instrumented with PMTs and is part of the active veto system. The LXe Skin is used to veto mostly gamma ray interactions within the TPC volume and is also sensitive to neutron scatters. The cryostat is surrounded near hermetically by ten acrylic vessels filled with Gadolinium-loaded liquid scintillator (GdLS). The GdLS is observed by 120 PMTs and stands within 238~t of DI water, which provides shielding to the detector and additional target material to veto atmospheric muons. The components and target media that surround the cryostat are described as the Outer Detector (OD). An in-depth review of the LZ experiment is presented in \autoref{chap:LZExperiment} alongside a brief overview of the author during the extensive time spent on-site during the course of this apprenticeship. 

The primary purpose of the OD is to detect neutron interactions that have coincident signals within the TPC. Neutrons in the MeV range are produced by radioactive decays via spontaneous fission and ($\alpha$,n) reactions in detector materials. These background neutrons will leave the TPC and interact further with the veto systems. As the neutron traverses the intervening material, it thermalizes and is captured on either the gadolinium or the hydrogen in the GdLS, or recoils off protons. Neutron recoils in the OD produce energy deposits of $\sim100~\text{keV}$ which correspond to pulses of $\sim15$ photons. The OD PMTs require calibration to single photon sensitivity to accurately measure these energy deposits by through-going particles. \autoref{chap:ODCommissioning} describes the single photoelectron response function used to calibrate the OD PMTs and, in turn, how the gain of the PMTs is reconstructed. The stability of the single photoelectron response and subsequent gain during the science runs is reported. A dedicated optical calibration is used to produce light to calibrate the OD PMTs single photoelectron response. The development of an \textit{in situ} monitoring system for the optical calibration system follows. Understanding the response of the OD PMTs and the stability of the single photoelectron signal during science runs is imperative to the performance of the veto detectors and their ability to effectively identify events which coincident signals in the TPC.

A WIMP discovery will require an excellent understanding of all background sources and the exclusion of all other possible explanations of the excess. The efficiency of the systems in the detection of radiation produced by the backgrounds should also be maximised whilst minimising the impact of the veto selection on the detector live time. \autoref{chap:VetoEfficiency} describes the series of improvements to the OD simulations that were made before the WS2024 science run. The refinement of the veto coincidence selection for the WS2024 science run is discussed and precedes the calculation used to estimate the veto system's ability to detect neutrons. The chapter concludes with a discussion on the impact that the veto selection and system has on the WS2024 analysis. 

In addition to neutrons produced through radioactive decays, atmospheric muons pose a threat to direct detection dark matter searches. Energetic neutrons, which could mimic a WIMP signal in the TPC, can be produced through muon interactions with the detector and surrounding material. Rare event search experiments are situated at deep underground sites to decrease the muon flux. To understand the effect that muon-induced backgrounds have on deep underground detectors, an initial measurement of the muon flux through the detector must be made. \autoref{chap:Muons} describes atmospheric muon production and their interactions as they travel from the upper Earth's atmosphere to the underground sites. The measurement of the muon flux using WS2022 and WS2024 data is presented alongside the diurnal muon modulation observed in the data set. When muons pass through the TPC, a series of S2-like and SE-like pulses associated with the track of ionisation are observed. A muon veto selection developed to remove events from the WIMP search data is outlined.

Finally, the world-leading results of the 2024 WIMP analysis are summarised in \autoref{chap:WS2024Result}. The author has made a considerable impact on this result through the continuous effort to calibrate the OD PMTs to single photoelectron sensitivity, the development of the muon veto, and the refinement of the veto coincidence selection and subsequent estimate of the veto system's neutron tagging efficiency.